\documentclass[12pt, a4paper]{article}
\usepackage[UTF8]{ctex}
\usepackage{enumerate}
\usepackage{amsmath}
\usepackage{amssymb}
\usepackage{blkarray}
\usepackage{geometry}
\geometry{left = 2.0cm, right = 2.0cm}

\begin{document}
\title{VE475 Intro to Cryptography Homework 3}
\author{Taoyue Xia, 518370910087}
\date{2021/06/08}
\maketitle

\section{Ex1--Finite Fields}
\begin{enumerate}
    \item Take X's value as 0, 1, 2 in $\mathbb{F}_3[X]$:
          $$0^2 + 1 = 1\ mod\ 3 \quad 1^2 + 1 = 2\ mod\ 3 \quad 2^2 + 1 = 2\ mod\ 3$$
          We can see that for $X \in \mathbb{F}_3[X]$, there doesn't exists an $X$ which makes $X^2 + 1 = 0\ mod\ 3$\newline
          Thus $X^2 + 1$ is irreducible in $\mathbb{F}_3[X]$.
    \item In question 1, we proved that $X^2 + 1$ is irreducible in $\mathbb{F}_3[X]$, 
          and the polynomial $1 + 2X$ 's degree is less than 2, according to the proof on page 39, c2, 
          Let $P(X) = X^2 + 1$, $A(X) = 1 + 2X$, then there always exists a $B(X)$, such that
          $A(X)\, B(X) = 1\ mod\ P(X)$, which means $B(x)$ is the multiplication inverse of $1 + 2X\ mod\ X^2 + 1$. 
          Proof done.
    \item Apply the extended Euclidean algorithm, let $a$ and $b$ be such that $a(1 + 2X) + b(X^2 + 1) = 1\ mod\ 3$.
          Then calculate in matrix form(a's value in the first column, b's value in the second):\newline
          $$
          \begin{pmatrix} 1 & 0 & 1 + 2X\\ 0 & 1 & X^2 + 1 \end{pmatrix}
          \Rightarrow \begin{pmatrix} 0 & 1 & X^2 + 1\\ 1 & 0 & 1 + 2X \end{pmatrix}
          \Rightarrow \begin{pmatrix} 1 & 0 & 1 + 2X\\ X & 1 & X + 1 \end{pmatrix}
          $$
          $$
          \Rightarrow \begin{pmatrix} X & 1 & X + 1\\ X + 1 & 1 & 2 \end{pmatrix}
          \Rightarrow \begin{pmatrix} X + 1 & 1 & 2\\ X^2 + 2X & X + 1 & 1 \end{pmatrix}
          $$
          Thus we can find that the multiplication inverse of $1 + 2X\ mod\ X^2 + 1$ is $X^2 + 2X$.

\end{enumerate}

\section{Ex2--AES}
\begin{enumerate}
    \item \begin{enumerate}[a)]
            \item The \emph{InvShiftRows} function cyclicly shift each row \emph{i}'s elements right for $i = 0,\ 1,\ 2,\ 3$.\newline
                  For example, if the $4\times 4$ matrix is 
                  $\begin{bmatrix} 
                        a_{00} & a_{01} & a_{02} & a_{03}\\
                        a_{11} & a_{12} & a_{13} & a_{10}\\
                        a_{22} & a_{23} & a_{20} & a_{21}\\
                        a_{33} & a_{30} & a_{31} & a_{32}
                  \end{bmatrix}$, then the matrix after the operation \emph{InvShiftRow} would be: 
                  $\begin{bmatrix} 
                        a_{00} & a_{01} & a_{02} & a_{03}\\
                        a_{10} & a_{11} & a_{12} & a_{13}\\
                        a_{20} & a_{21} & a_{22} & a_{23}\\
                        a_{30} & a_{31} & a_{32} & a_{33}
                  \end{bmatrix}$.
            \item The inverse of \emph{AddRoundKey} will make the 4 32-bits words xor with the expansion key, for example, 
                  in AES-128, for 11 rounds in the reverse order. This is because anything xor a value twice would keep unchanged.
            \item The $4\times 4$ matrix used for \emph{MixColumns} is (in hexadecimal form):
                  $$A_1 = 
                  \begin{pmatrix}
                        00000010 & 00000011 & 00000001 & 00000001\\
                        00000001 & 00000010 & 00000011 & 00000001\\
                        00000001 & 00000001 & 00000010 & 00000011\\
                        00000011 & 00000001 & 00000001 & 00000010
                  \end{pmatrix} = \begin{pmatrix} 02 & 03 & 01 & 01\\ 01 & 02 & 03 & 01\\ 01 & 01 & 02 & 03\\ 03 & 01 & 01 & 02\end{pmatrix}
                  $$
                  The $4\times 4$ matrix used for \emph{InvMixColumns} is given by:
                  $$A_2 = 
                  \begin{pmatrix}
                        00001110 & 00001011 & 00001101 & 00001001\\
                        00001001 & 00001110 & 00001011 & 00001101\\
                        00001101 & 00001001 & 00001110 & 00001011\\
                        00001011 & 00001101 & 00001001 & 00001110
                  \end{pmatrix} = \begin{pmatrix} 0e & 0b & 0d & 09\\ 09 & 0e & 0b & 0d\\ 0d & 09 & 0e & 0b\\ 0b & 0d & 09 & 0e\end{pmatrix}
                  $$
                  Then we calculate $A_2\times A_1$ in $\mathbb{GF}(2^8)$, for example, the four elements of the first row of 
                  matrix $B = A_2 \times A_1$ will be:
                  \begin{equation*}
                        \begin{split}
                              B_{0,1} = (0e\cdot 02) \oplus (0b\cdot 01) \oplus (0d\cdot 01) \oplus (09\cdot 03) = 01_{16}\\
                              B_{0,2} = (0e\cdot 03) \oplus (0b\cdot 02) \oplus (0d\cdot 01) \oplus (09\cdot 01) = 00_{16}\\
                              B_{0,3} = (0e\cdot 01) \oplus (0b\cdot 03) \oplus (0d\cdot 02) \oplus (09\cdot 01) = 00_{16}\\
                              B_{0,4} = (0e\cdot 01) \oplus (0b\cdot 01) \oplus (0d\cdot 03) \oplus (09\cdot 02) = 00_{16}
                        \end{split}
                  \end{equation*}
                  Using the same method, we can calculate the remaining three rows of $B$, finally (in hexadecimal form):
                  $$ B = 
                  \begin{pmatrix}
                        01 & 00 & 00 & 00\\
                        00 & 01 & 00 & 00\\
                        00 & 00 & 01 & 01\\
                        00 & 00 & 00 & 01
                  \end{pmatrix} = \mathbb{I}_4
                  $$
                  Thus it is the reason why the transformation of \emph{InvMixColumns} is given by the multiplication by matrix $A_2$. 
            \end{enumerate} 
      \item Firstly, we apply the key expansion schedule to generate round keys according to the original key. 
            Then we perform \emph{AddRoundKey} with $round\_keys[40 \sim 43]$, each has one 32-bit word.\newline
            In the next nine rounds, we perform \emph{InvShiftRows}, \emph{InvSubBytes}, \emph{AddRoundKey} 
            (with $round\_keys[40-4i \sim 40-4(i-1)]$ for $i \in [1, 9]$), and \emph{InvMixColumns} in order.\newline
            Finally, We perform the last round of \emph{InvShiftRows}, \emph{InvSubBytes}, \emph{AddRoundKey} 
            with $round\_keys[0 \sim 3]$, then we can get the original plaintext from the ciphertext.
      \item In process of \emph{InvShiftRows}, we just change the position of some elements without changing any value. 
            In process of \emph{InvSubBytes}, we look up the corresponding value of each element in the inverse s-box, 
            and substitute the original value. Therefore, it doesn't matter in which order these two processes are performed. 
            That's why they can be applied on reverse order.
      \item \begin{enumerate}[a)]
                  \item As \emph{AddRoundKey} will perform an xor betweeen treated text and the key in different rounds, 
                        the value of each 32-bit word would probably change. What's more, in process \emph{InvMixColumns}, 
                        multiplication and addition in Galois field $\mathbb{GF}(2^8)$ are performed, 
                        which can also change the values of words. As addition and multiplication with different values 
                        can lead to a completely distinct result, thus the order of application of \emph{AddRoundKey} 
                        and \emph{InvMixColums} cannot be reversed.
                  \item $$((m_{i,j})\ (a_{i,j})) \oplus (k_{i,j})$$
                  \item As the initial matrix is $(a_{i,j})$, from the above order, we can get:
                        \begin{align*}
                              (a_{i,j}) &= (m_{i,j})^{-1} ((e_{i,j}) \oplus (k_{i,j}))\\
                                        &= (m_{i,j})^{-1} (e_{i,j}) \oplus (m_{i,j})^{-1} (k_{i,j})
                        \end{align*}
                        So the inverse operation is:
                        $$(e_{i, j})\ \rightarrow \ (m_{i,j})^{-1} (e_{i,j}) \oplus (m_{i,j})^{-1} (k_{i,j})$$
                  \item The \emph{InvAddRoundKey} operation will first calculate the multiplication of the \emph{InvMatrix} 
                        and the key of the corresponding round, then perform an xor with the text which has been operated 
                        by the \emph{InvMixColumns} method.
            \end{enumerate}
      \item Firstly, we apply the key expansion schedule to generate round keys according to the original key. 
            Then we perform \emph{AddRoundKey} with $round\_keys[40 \sim 43]$, each has one 32-bit word.\newline
            In the next nine rounds, we perform \emph{InvSubBytes}, \emph{InvShiftRows}, \emph{InvMixColumns}, 
            and \emph{InvAddRoundKey} (with $round\_keys[40-4i \sim 40-4(i-1)]$ for $i \in [1, 9]$) in order.\newline
            Finally, We perform the last round of \emph{InvSubBytes}, \emph{InvShiftRows}, \emph{AddRoundKey} 
            with $round\_keys[0 \sim 3]$, then we can get the original plaintext from the ciphertext.
      \item The advantage is that the order of non-inverse operations and inverse operations are the same, 
            so that it's easier to understand and implement.
\end{enumerate}

\newpage
\section{Ex3--DES}
\begin{enumerate}
      \item In DES, the size of input text and key are both 64 bits.
            \begin{enumerate}[a)]
                  \item The input plaintext is enciphered by the following permutation table \textbf{\emph{IP}}:
                        \begin{center}
                              \begin{tabular}{cccccccc}
                                    58 & 50 & 42 & 34 & 26 & 18 & 10 & 2\\
                                    60 & 52 & 44 & 36 & 28 & 20 & 12 & 4\\
                                    62 & 54 & 46 & 38 & 30 & 22 & 14 & 6\\
                                    64 & 56 & 48 & 40 & 32 & 24 & 16 & 8\\
                                    57 & 49 & 41 & 33 & 25 & 17 & 9  & 1\\
                                    59 & 51 & 43 & 35 & 27 & 19 & 11 & 3\\
                                    61 & 53 & 45 & 37 & 29 & 21 & 13 & 5\\
                                    63 & 55 & 47 & 39 & 31 & 23 & 15 & 7
                              \end{tabular}
                        \end{center}
                        That is, in the enciphered 64 bits, the first bit is the original 58th bit, the second is the 50th, etc.\newline
                  \item Then the key is reduced to 56 bits in the same way as above, looking up value in a table, 
                        and replace the original bit.
                  \item Then the 64-bit enciphered text is divided into two 32-bit content $L_0$ and $R_0$. Define a operation function $f$, 
                        we can calculate:
                        $$L_1 = R_0$$
                        $$R_1 = L_0 \oplus f(R_0, K_0)$$
                        Repeat the process for 16 rounds, we will get final $L_{16}$ and $R_{16}$.
                  \item For the operation function $f$, we first extend $R_i,\ (i \in [0,\ 16])$ from 32 bits to 48 bits. 
                        The method is like that in (b), in a look-up table, generating each bit from the input $R_i$.\newline
                        Then the 56-bit key will be splitted into two 28-bit keys, and each shift left for 1 or 2 bits according to 
                        each round, the table is shown below:
                        \begin{center}
                              \begin{tabular}{c|cccccccccccccccc}
                                    round & 1 & 2 & 3 & 4 & 5 & 6 & 7 & 8 & 9 & 10 & 11 & 12 & 13 & 14 & 15 & 16\\
                                    \hline
                                    shift & 1 & 1 & 2 & 2 & 2 & 2 & 2 & 2 & 1 & 2  & 2  & 2  & 2  & 2  & 2  & 1
                              \end{tabular}
                        \end{center}
                        After that, combine the two parts into a 56-bit key, and reduce it o 48 bits in the same way as (b), 
                        with a different look-up table.\newline
                        Then we xor the extended 48-bit $R_i$ and the key, denote the output as $X$. $X$ is then divided into 
                        8*6 bits, each 6 bits will pass a S-box $S_i,\ (i \in [1,\ 8])$, and each output a 4-bit content.\newline
                        Finally, join the 8 groups of 4 bits, we will get $f(R_i, K_i)$ which is 32 bits.
                  \item After the 16 rounds of transforming, we can get a 64-bit data. Finally, we will perform an inverse of 
                        the initial permutation, by the following table \textbf{\emph{$IP^{-1}$}}:
                        \begin{center}
                              \begin{tabular}{cccccccc}
                                    40 & 8 & 48 & 16 & 56 & 24 & 64 & 32\\
                                    39 & 7 & 47 & 15 & 55 & 23 & 63 & 31\\
                                    38 & 6 & 46 & 14 & 54 & 22 & 62 & 30\\
                                    37 & 5 & 45 & 13 & 53 & 21 & 61 & 29\\
                                    36 & 4 & 44 & 12 & 52 & 20 & 60 & 28\\
                                    35 & 3 & 43 & 11 & 51 & 19 & 59 & 27\\
                                    34 & 2 & 42 & 10 & 50 & 18 & 58 & 26\\
                                    33 & 1 & 41 & 9  & 49 & 17 & 57 & 25
                              \end{tabular}
                        \end{center}
            \end{enumerate}
            Then the whole encryption is done.\newline
            For decryption, we just need to reverse the key's order and apply the reverse of each operation.
      \item Linear cryptanalysis:\newline
            There are two parts for linear cryptanalysis. 
            The first is to construct linear equations relating plaintext, ciphertext and key bits that have a high bias. 
            The second is to use these linear equations in conjunction with known plaintext-ciphertext pairs to derive key bits.\newline
            \newline
            Differential cryptanalysis:\newline
            Differential cryptanalysis is usually a chosen plaintext attack. 
            The basic method uses pairs of plaintext related by a constant difference. 
            Difference can be defined in several ways, but the eXclusive OR (XOR) operation is usual.\newline
            The attack relies primarily on the fact that a given input/output difference pattern only occurs for certain values of inputs.
            Usually the attack is applied in essence to the non-linear components as if they were a solid component 
            (usually they are in fact look-up tables or S-boxes). Observing the desired output difference 
            (between two chosen or known plaintext inputs) suggests possible key values.
      \item Triple DES:\newline
            We define the encryption of DES as $E_K(P)$, where $K$ stands for key, $P$ stands for original plaintext, 
            and the decryption of DES as $D_K(C)$, where $C$ is the ciphertext.\newline
            Triple DES, also knows are the TDEA, encode the plaintext for three rounds of DES to get the ciphertext:
            $$C = E_{K_3}(E_{K_2}(E_{K_1}(P)))$$
            Then the decode process is the inverse:
            $$P = D_{K_1}(D_{K_2}(D_{K_3}(C)))$$
            The standard for the three keys $K_1,\ K_2,\ K_3$ is defined below:
            \begin{enumerate}[(1)]
                  \item Key Option 1: $K_1$, $K_2$ and $K_3$ are three independent keys.
                  \item Key Option 3: $K_1$ and $K_2$ are independent keys, and $K_3 = K_1$.
                  \item Key Option 3: $K_1 = K_2 = K_3$.
            \end{enumerate}   
            Meet-in-the-middle attack is the reason why Double DES is replaced by Triple DES. It's logic is:
            \begin{align*}
                  C &= E_{K_2}(E_{K_1}(P))\\
                  D_{K_2}(C) &= D_{K_2}(E_{K_2}(E_{K_1}(P)))\\
                  D_{K_2}(C) &= E_{K_1}(P)
            \end{align*}
            The attacker can compute $E_{K_1}(P)$ for all possible values of $K_1$ and $D_{K_2}(C)$ for all 
            possible values of $K_2$ for a total of $2^{K_1} + 2^{K_2}$ operations.\newline
            As Double DES uses two keys $K_1$ and $K_2$ in same size 56 bits, attackers can bruteforce Double DES 
            in $2^{57}$ operations and $2^{56}$ space to get the keys, which is not safe at all.\newline
            However, for Triple DES, attackers need up to $2^{K_1+K_2} + 2^{K_3}$, namely $2^{112}$ operations 
            and $2^{56}$ space to bruteforce and get the keys, which is safe, but still not secure.\newline
            This is the reason why Triple DES is used instead of Double DES.
      \item Traditionally, the \textbf{\emph{crypt()}} function which encrypts users' passwords by DES, 
            and save the encoded content in /etc/passwd. However, DES is proved to be not safe nowadays, 
            so modern Unix systems use \textbf{SHA-256} and \textbf{MD5}, which is more secure. 
            So if one doesn't directly show his password file to others, it is almost impossible to 
            cause password leak.
            
\end{enumerate}

\section{Ex4--Programming}
The codes and makefile are attached in folder \textbf{ex4}, with a README file.

\end{document}