\documentclass[12pt, a4paper]{article}
\usepackage[UTF8]{ctex}
\usepackage{enumerate}
\usepackage{amsmath}
\usepackage{amssymb}
\usepackage{blkarray}
\usepackage{geometry}

\geometry{left = 2.0cm, right = 2.0cm}

\begin{document}
\title{VE475 Intro to Cryptography Homework 10}
\author{Taoyue Xia, 518370910087}
\date{2021/07/27}
\maketitle

\section*{Ex1 --- Group structure on an elliptic curve}
Taking two points $P_1 = (x_1, y_1)$, $P_2 = (x_2, y_2)$, let $P_3 = P_1 + P_2 = (x_3, y_3)$, 
where $x_3 = (m^2 - x_1 - x_2)$ and $y_3 = m(x_1 - x_3) - y_1$. We first prove that for $P_3$, 
it also satisfy $y_3^2 = x_3^3 + bx_3 + c$ for different $m$ values. 
\begin{align*}
    x_3^3 + bx_3 + c &= (m^2 - x_1 - x_2)^3 + b(m^2 - x_1 - x_2) + c\\
                     &= m^6 - 3m^4(x_1 + x_2) + 3m^2(x_1^2 + x_2^2) + 6m^2x_1x_2 \\
                     &\quad - 3(x_1^2x_2 + x_2^2x_1) - x_1^3 - x_2^3 + b(m^2 - x_1 - x_2) + c\\
                     \\
            y_3^2    &= [m(x_1 - x_3) - y_1]^2\\
                     &= [m(2x_1 + x_2 - m^2) - y_1]^2\\
                     &= m^2(2x_1 + x_2 - m^2)^2 - 2m(2x_1 + x_2 - m^2)y_1 + y_1^2\\
                     &= m^6 - 4m^4x_1 - 2m^4x_2 + 4m^2x_1^2 + 4m^2x_1x_2 + m^2x_2^2 - 4mx_1y_1\\
                     &\quad - 2mx_2y_1 + 2m^3y_1 + y_1^2\\
                     \\
    y_3^2 - (x_3^3 + bx_3 + c) &= -m^4(x_1 - x_2) + m^2x_1^2 - 2m^2x_2^2 - 2m^2x_1x_2 - 4mx_1y_1 - 2mx_2y_1\\
                               &\quad + 2m^3y_1 + y_1^2 + 3(x_1^2x_2 + x_2^2x_1) + x_1^3 + x_2^3 - b(m^2 - x_1 - x_2) - c
\end{align*}
When $P_1 \neq P_2$, $m = \frac{y_2 - y_1}{x_2 - x_1}$, thus
\begin{align*}
    y_3^2 - (x_3^3 + bx_3 + c) &= \frac{(y_2 - y_1)^4}{(x_2 - x_1)^3} + \frac{(y_2 - y_1)^3}{(x_2 - x_1)^3} y_1
                                  + \frac{(y_2 - y_1)^2}{(x_2 - x_1)^2}(x_1^2 - 2x_2^2 - 2x_1x_2 - b)\\
                               &\quad -\frac{y_2 - y_1}{x_2 - x_1}(4x_1y_1 + 2x_2y_1) + y_1^2 + 3(x_1^2x_2 + x_2^2x_1)\\
                               &\quad + x_1^3 + x_2^3 + b(x_1 + x_2) - c
\end{align*}
After some calculation, we can prove that $y_3^2 - (x_3^3 + bx_3 + c) = 0$. However, this part is too tedious, 
so it will not be illustrated.\\
Then when $P_1 = P_2$, indicating $x_1 = x_2,\ y_1 = y_2$, with $m = (3x_1^2 + b) / (2y_1)$, we can get:
\begin{align*}
    y_3^2 - (x_3^3 + bx_3 + c) &= -m^4(x_1 - x_2) + m^2x_1^2 - 2m^2x_2^2 - 2m^2x_1x_2 - 4mx_1y_1 - 2mx_2y_1\\
                               &\quad + 2m^3y_1 + y_1^2 + 3(x_1^2x_2 + x_2^2x_1) + x_1^3 + x_2^3 - b(m^2 - x_1 - x_2) - c\\
                               &= -3(\frac{3x_1^2+b}{2y_1})^2x_1^2 - 6\frac{3x_1^2+b}{2y_1}x_1y_1 + 2(\frac{3x_1^2+b}{2y_1})^3y_1 
                                  + y_1^2 + 8x_1^3\\
                               &\quad - b(\frac{3x_1^2+b}{2y_1})^2 + 2bx_1 - c\\
                               &= \frac{-3x_1^2(9x_1^4 + 6x_1^2b + b^2) + (27x_1^6 + 27x_1^4b + 9x_1^2b^2 + b^3) - b(9x_1^4 + 6x_1^2b + b^2)}{4y_1^2}\\
                               &\quad - (9x_1^2 + 3b)x_1 + (x_1^3 + bx_1 + c) + 8x_1^3 + 2bx_1 -c\\
                               &= 0
\end{align*}
Therefore, we also prove that when $P_1 = P_2$, the addition also holds. So the addition law over $E$ is proved.\newline
\newline
Then we will prove the commutativity, which is for $P_1, P_2 \in E,\ P_1 + P_2 = P_2 + P_1$. When $P_1 = P_2$, 
it is always true, so we just need to consider the case $P_1 \neq P_2$. Let $P_1 = (x_1, y_1)$, $P_2 = (x_2, y_2)$, 
and $P_1 + P_2 = P_3 = (x_3, y_3)$, $P_2 + P_1 = P_3^\prime = (x_3^\prime, y_3^\prime)$. With $m = \frac{y_2-y_1}{x_2-x_1}$, we can have:
$$x_3 = m^2 - x_1 - x_2 = m^2 - x_2 - x_1 = x_3^\prime$$
$$y_3 = m(x_1 - x_3) - y_1 = \frac{(y_2 - y_1)(2x_1 + x_2 - m^2)}{x_2 - x_1} - y_1$$
$$y_3^\prime = m(x_1 - x_3^\prime) - y_1 = \frac{(y_2 - y_1)(2x_1 + x_2 - m^2)}{x_2 - x_1} - y_1$$
Therefore, we can see that $x_3 = x_3^\prime$ and $y_3 = y_3^\prime$, thus $P_3 = P_3^\prime$. Therefore, 
the commutativity is proved.\newline
\newline
Then we will prove the associativity. For $\mathcal{O},\ P_1,\ P_2,\ P_3$ on $E$, according to the properties of elliptic curves, 
we can know that $P_1 + P_2,\ -(P_1 + P_2),\ P_2 + P_3,\ -(P_2 + P_3)$ are also on the curve $E$. 
Then we can see that the point $-(P_1 + P_2 + P_3)$ is on the line which passes through points $P_1 + P_2$ and $P_3$. 
Also, the point is on the line which passes through points $P_1$ and $P_2 + P_3$. Which gives 
$$(P_1 + P_2) + P_3 + (-(P_1 + P_2 + P_3)) = 0 \qquad \text{and} \qquad P_1 + (P_2 + P_3) + (-(P_1 + P_2 + P_3)) = 0$$
Therefore, we have proved that $(P_1 + P_2) + P_3 = P_1 + (P_2 + P_3)$. Therefore, the assosciativity is proved.\\
The whole proposition is proved.

\section*{Ex2 --- Number of points on an elliptic curve}
\begin{enumerate}
    \item We have $E$ as $y^2 = x^3 + 3x + 7$ over $\mathbb{F}_{11}$ and $P = (8, 9)$. 
          Since $[2]P = P + P = (x_2, y_2)$, we can calculate it as:
          $$m \equiv \frac{3\cdot 8^2 + 3}{2\cdot 9} \equiv \frac{65}{6} \equiv 65\cdot 2\equiv 9\ mod\ 11$$
          $$x_2 \equiv m^2 - 2\cdot 8 \equiv 81 - 16 \equiv 10\ mod\ 11$$
          $$y_2 \equiv m(8 - 10) - 9 \equiv -18 - 9 \equiv 6\ mod\ 11$$
          So we can see that $[2]P = (10, 6)$\newline
          For $[5]P$, we can calculate it as $[5]P = [2]P + [2]P + P$. we first calculate $[4]P = [2]P + [2]P = (x_4, y_4)$:
          $$m_4 \equiv \frac{3\cdot 10^2 + 3}{2\cdot 6} \equiv \frac{101}{4} \equiv 101\cdot 3 \equiv 6\ mod\ 11$$
          $$x_4 \equiv m_4^2 - 2x_2 \equiv 36 - 20 \equiv 5\ mod\ 11$$
          $$y_4 \equiv m_4(x_2 - x_4) - y_2 \equiv 6\cdot (10 - 5) - 6 \equiv 2\ mod\ 11$$
          So we get that $[4]P = (5, 2)$. Then we can calculate $[5]P = [4]P + P = (x_5, y_5)$:
          $$m_5 \equiv \frac{y_1 - y_4}{x_1 - x_4} \equiv \frac{9-2}{8-5} \equiv 7\cdot 4 \equiv 6\ mod\ 11$$
          $$x_5 \equiv m_5^2 - x_4 - x_1 \equiv 36 - 5 - 8 \equiv 1\ mod\ 11$$
          $$y_5 \equiv m_5(x_4 - x_5) - y_4 \equiv 6\cdot (5-1) - 2 \equiv 22 \equiv 0\ mod\ 11$$
          Therefore, we see that $[5]P = (1, 0)$.\newline
          For $[10]P$, we see that $[10]P = [5]P + [5]P$. However, we find that the $y$ value for $[5]P$ is 0, 
          so the line through $[5]P$ is vertical to the $x$ axis. Therefore, $[10]P = \mathcal{O}$, the unit element.
    \item We will show by the following table. 
          \begin{center}
              \begin{tabular}{c|c|c|c}
                  \hline
                  $x\ \text{mod}\ 11$ & $y^2\ \text{mod}\ 11$ & $y\ \text{mod}\ 11$ & Points on $E$\\
                  \hline
                  0 & 7 & & \\
                  1 & 0 & 0 & (1,0)\\
                  2 & 10 & & \\
                  3 & 10 & & \\
                  4 & 6 & & \\
                  5 & 4 & 2 or 9 & (5,2), (5,9)\\
                  6 & 10 & & \\
                  7 & 8 & & \\
                  8 & 4 & 2 or 9 & (8,2), (8,9)\\
                  9 & 4 & 2 or 9 & (9,2), (9,9)\\
                  10 & 3 & 5 or 6 & (10,5), (10,6)\\
                  \hline
              \end{tabular}
          \end{center}
          So there are $1+4\cdot 2+1 = 10$ points on $E$, containing the unit element $\mathcal{O}$. 
    \item The points are (1,0), (5,2), (5,9), (8,2), (8,9), (9,2), (9,9), (10,5), (10,6), and a unit element $\mathcal{O} = \infty$.
\end{enumerate}

\section*{Ex3 --- ECDSA}
For the Elliptic Curve Digital Signature Algorithm (ECDSA), Alice and Bob need the following parameters:
\begin{enumerate}
    \item An elliptic curve $E$, and the equation used.
    \item A base point $G$ on $E$, which generates a subgroup of large prime order $n$.
    \item The prime order $n$, which means that $[n]G = \mathcal{O}$, the unit element.
    \item A randomly selected private key $d_A$ in the interval $[1, n-1]$.
    \item A public key $Q_A$, calculated by the elliptic curve $Q_A = [d_A]G$.
    \item The message $m$ to send.
\end{enumerate}
We first define a cryptographic hash function $H$, and the output converted to integer. 
For Alice to sign the message, she follows the steps:
\begin{enumerate}
    \item Calculate $e = H(m)$.
    \item Set $z$ to be the $L_n$ leftmost bits of e, where $L_n$ represents for the bit length of $n$.
    \item Select a cryptographic secure random number $k$ in the interval $[1,n-1]$. 
    \item Calculate the curve point $(x_1, y_1) = [k]G$.
    \item Calculate $r = x_1\ \text{mod}\ n$. If $r = 0$, go back to step 3.
    \item Calculate $s = k^{-1}(z + rd_A)\mod n$. If $s = 0$, go back to step 3.
    \item The signature is the pair $(r,s)$.
\end{enumerate}
Note that we need to make $k$ secret, as well as select different $k$ for different signatures. 
Otherwise step 6 can be used to solve $d_A$, the private key. The procedures are as follows:\\
Given two signatures $(r,s)$ and $(r,s^\prime)$, which uses the same $k$ for different messages $m$ and $m^\prime$, 
an attacker can calculate $z$ and $z^\prime$, and since $s - s^\prime \equiv k^{-1}(z - z^\prime)\mod n$, 
the attacker can find $k \equiv \frac{z - z^\prime}{s - s^\prime}\mod n$. Since $s \equiv k^{-1}(z + rd_A)\mod n$, 
the attacker can now calculate $d_A \equiv \frac{ks - z}{r}\mod n$. Therefore, using the same $k$ multiple times is dangerous.\newline
Then for Bob to authenticate Alice's signature, he must have a copy of her public-key curve point $Q_A$. 
Bob can first verify $Q_A$ as a valid point as follows:
\begin{enumerate}
    \item Check that $Q_A$ is not equal to the unit element $\mathcal{O}$.
    \item Check that $Q_A$ lies on the curve.
    \item Check that $[n]Q_A = \mathcal{O}$.
\end{enumerate}
After verifying $Q_A$, Bob can verify the signature following these steps:
\begin{enumerate}
    \item Verify that $r$ and $s$ are integers in $[1,n-1]$. If not, the signature is invalid.
    \item calculate $e = H(m)$, with the same hash function $H$.
    \item Let $z$ be the $L_n$ leftmost bits of $e$.
    \item Calculate $u_1 \equiv zs^{-1}\mod n$, $u_2 \equiv rs^{-1}\mod n$.
    \item Calculate the curve point $(x_1, y_1) = [u_1]G + [u_2]Q_A$. If the result is $\mathcal{O}$, then the signature is invalid.
    \item Finally, the signature is valid if $r\equiv x_1\mod n$, invalid otherwise.
\end{enumerate}
Then we show the correctness of such algorithm, denote $C$ as the curve point calculated in step 5.
\begin{align*}
    C &= [u_1]G + [u_2]Q_A\\
      &= [u_1]G + [u_2d_A]G\\
      &= [u_1 + u_2d_A]G\\
      &= [zs^{-1} + rs^{-1}d_A]G\\
      &= [s^{-1}(z + rd_A)]G\\
      &= [k]G = (x_1, y_1)
\end{align*}
Therefore, the point corresponds to the point generated by Alice, the correctness is verified.\newline
The benefits of ECDSA is shown below:
\begin{enumerate}
    \item It needs smaller sized keys to meet the requirement of a specific security level. Namely, 
          if the security level is 128-bit, its key just needs two times the bit length, which is 256 bits, to be secure. 
          However, for common RSA it should be 3072 bits.
    \item Smaller key means faster calculation and faster transmission. The signature needs less time to create, 
          less space to store, and less time to transmit to the receiver.
\end{enumerate}

\section*{Ex4 --- BB84}
BB84 is a quantum key distribution scheme developed in 1984, which is the first quantum cryptography protocol.\newline
In the BB84 scheme,  Alice wishes to send a private key to Bob. She begins with two strings of bits, $a$ and $b$, each $n$ bits long. 
She then encodes these two strings as a tensor product of $n$ qubits:
$$|\psi \rangle = \bigotimes_{i=1}^n |\psi_{a_ib_i} \rangle$$
Where $a_i$, $b_i$ are the $i_{th}$ bits of $a$ and $b$. Together, $a_ib_i$ give us an index into the following four qubit states:
$$|\psi_{00} \rangle = |0\rangle$$
$$|\psi_{10} \rangle = |1\rangle$$
$$|\psi_{01} \rangle = |+\rangle = \frac{1}{\sqrt{2}}|0\rangle + \frac{1}{\sqrt{2}}|1\rangle$$
$$|\psi_{11} \rangle = |-\rangle = \frac{1}{\sqrt{2}}|0\rangle - \frac{1}{\sqrt{2}}|1\rangle$$
Alice sends $|\psi \rangle$  over a public and authenticated quantum channel $\mathcal{E}$ to Bob. 
Bob receives a state $\mathcal{E}(\rho) = \mathcal{E}(|\psi \rangle \langle \psi |)$, 
where ${\mathcal {E}}$ represents both the effects of noise in the channel and eavesdropping by a third party Eve. 
However, since only Alice knows $b$, it makes it virtually impossible for either Bob or Eve to distinguish the states of the qubits. 
Also, after Bob has received the qubits, we know that Eve cannot be in possession of a copy of the qubits sent to Bob, 
by the no-cloning theorem, unless she has made measurements. Her measurements, however, 
risk disturbing a particular qubit with probability $1/2$ if she guesses the wrong basis.\newline
Bob proceeds to generate a string of random bits $b^\prime$ of the same length as $b$, 
and then measures the qubits he has received from Alice, obtaining a bit string $a^\prime$. At this point, 
Bob announces publicly that he has received Alice's transmission. Alice then knows she can now safely announce $b$. 
Bob communicates over a public channel with Alice to determine which $b_i$ and $b^\prime_i$ are not equal. 
Both Alice and Bob now discard the bits in $a$ and $a^\prime$ where $b$ and $b^\prime$ do not match.

From the remaining $k$ bits where both Alice and Bob measured in the same basis, 
Alice randomly chooses $k/2$ bits and discloses her choices over the public channel. 
Both Alice and Bob announce these bits publicly and run a check to see whether more than a certain number of them agree. 
If this check passes, 
Alice and Bob proceed to use information reconciliation and privacy amplification techniques to create some number of shared secret keys. 
Otherwise, they cancel and start over.

\section*{Ex5 --- Quantum key distribution}
\begin{enumerate}
    \item Alice and Bob can use the quantum channel to generate and share a private key, 
          and then use the classical channel to tranfer encrypted messages with that key.
    \item If Eve observes photons in the quantum channel, the original state will collapse and change to another one. 
          This change can be easily detected by Alice and Bob, since they are transmitting information. Therefore, 
          they can simply choose another private key, and Eve cannot get any information.
\end{enumerate} 

\section*{Ex6 --- Simple questions}
\begin{enumerate}
    \item First show the expression of $U_1\otimes V_1$, denote the elements as $u_{1_{ij}}, v_{1_{ij}}$:
          \begin{align*}
            U_1 \otimes V_1 &= 
            \begin{pmatrix}
                u_{1_{11}}V_1 & u_{1_{12}}V_1 & \cdots & u_{1_{1n}}V_1\\
                u_{1_{21}}V_1 & u_{1_{22}}V_1 & \cdots & u_{1_{2n}}V_1\\
                \vdots & \vdots & \ddots & \vdots\\
                u_{1_{n1}}V_1 & u_{1_{n2}}V_1 & \cdots & u_{1_{nn}}V_1\\
            \end{pmatrix}\\
            &= \begin{pmatrix}
                u_{1_{11}}v_{1_{11}} & u_{1_{11}}v{1_{12}} & \cdots & u_{1_{11}}v{1_{1n}} & u_{1_{12}}v_{1_{11}} & \cdots & u_{1_{1n}}v_{1_{1n}}\\
                u_{1_{11}}v_{1_{21}} & u_{1_{11}}v{1_{22}} & \cdots & u_{1_{11}}v{1_{2n}} & u_{1_{12}}v_{1_{21}} & \cdots & u_{1_{1n}}v_{1_{2n}}\\
                \vdots & \vdots & \ddots & \vdots & \vdots & \ddots & \vdots\\
                u_{1_{11}}v_{1_{n1}} & u_{1_{11}}v{1_{n2}} & \cdots & u_{1_{11}}v{1_{nn}} & u_{1_{12}}v_{1_{n1}} & \cdots & u_{1_{1n}}v_{1_{nn}}\\
                u_{1_{21}}v_{1_{11}} & u_{1_{21}}v{1_{12}} & \cdots & u_{1_{21}}v{1_{1n}} & u_{1_{22}}v_{1_{11}} & \cdots & u_{1_{2n}}v_{1_{1n}}\\
                \vdots & \vdots & \ddots & \vdots & \vdots & \ddots & \vdots\\
                u_{1_{n1}}v_{1_{n1}} & u_{1_{n1}}v{1_{n2}} & \cdots & u_{1_{n1}}v{1_{nn}} & u_{1_{n2}}v_{1_{n1}} & \cdots & u_{1_{nn}}v_{1_{nn}}\\
            \end{pmatrix}
          \end{align*}
          Then we can calculate $U_1U_2$.
          $$U_1U_2 = 
          \begin{pmatrix}
              \sum_{i=1}^n u_{1_{1i}}u_{2_{i1}} & \sum_{i=1}^n u_{1_{1i}}u_{2_{i2}} & \cdots & \sum_{i=1}^n u_{1_{1i}}u_{2_{in}}\\
              \sum_{i=1}^n u_{1_{2i}}u_{2_{i1}} & \sum_{i=1}^n u_{1_{2i}}u_{2_{i2}} & \cdots & \sum_{i=1}^n u_{1_{2i}}u_{2_{in}}\\
              \vdots & \vdots & \ddots & \vdots\\
              \sum_{i=1}^n u_{1_{ni}}u_{2_{i1}} & \sum_{i=1}^n u_{1_{ni}}u_{2_{i2}} & \cdots & \sum_{i=1}^n u_{1_{ni}}u_{2_{in}}\\
          \end{pmatrix}
          $$
          Similarly, we can use the same method to express $U_2\otimes V_2$ and $V_1V_2$. In this way, 
          $(U_1\otimes U_2)\cdot(U_2\otimes V_2)$ can be shown as:
          $$
          \begin{pmatrix}
              \sum_{i=1}^n \sum_{j=1}^n u_{1_{1i}}v_{1_{1j}}u_{2_{i1}}v_{2_{j1}} & \sum_{i=1}^n \sum_{j=1}^n u_{1_{1i}}v_{1_{1j}}u_{2_{i1}}v_{2_{j2}} & \cdots & \sum_{i=1}^n \sum_{j=1}^n u_{1_{1i}}v_{1_{1j}}u_{2_{in}}v_{2_{jn}}\\
              \sum_{i=1}^n \sum_{j=1}^n u_{1_{1i}}v_{1_{2j}}u_{2_{i1}}v_{2_{j1}} & \sum_{i=1}^n \sum_{j=1}^n u_{1_{1i}}v_{1_{2j}}u_{2_{i1}}v_{2_{j2}} & \cdots & \sum_{i=1}^n \sum_{j=1}^n u_{1_{1i}}v_{1_{2j}}u_{2_{in}}v_{2_{jn}}\\
              \vdots & \vdots & \ddots & \vdots\\
              \sum_{i=1}^n \sum_{j=1}^n u_{1_{ni}}v_{1_{nj}}u_{2_{i1}}v_{2_{j1}} & \sum_{i=1}^n \sum_{j=1}^n u_{1_{ni}}v_{1_{nj}}u_{2_{i1}}v_{2_{j2}} & \cdots & \sum_{i=1}^n \sum_{j=1}^n u_{1_{ni}}v_{1_{nj}}u_{2_{in}}v_{2_{jn}}\\
          \end{pmatrix}
          $$
          Then $(U_1U_2)\otimes (V_1V_2)$ can be expressed as:
          $$
          \begin{pmatrix}
            \sum_{i=1}^n u_{1_{1i}}u_{2_{i1}}V_1V_2 & \sum_{i=1}^n u_{1_{1i}}u_{2_{i2}}V_1V_2 & \cdots & \sum_{i=1}^n u_{1_{1i}}u_{2_{in}}V_1V_2\\
            \sum_{i=1}^n u_{1_{2i}}u_{2_{i1}}V_1V_2 & \sum_{i=1}^n u_{1_{2i}}u_{2_{i2}}V_1V_2 & \cdots & \sum_{i=1}^n u_{1_{2i}}u_{2_{in}}V_1V_2\\
            \vdots & \vdots & \ddots & \vdots\\
            \sum_{i=1}^n u_{1_{ni}}u_{2_{i1}}V_1V_2 & \sum_{i=1}^n u_{1_{ni}}u_{2_{i2}}V_1V_2 & \cdots & \sum_{i=1}^n u_{1_{ni}}u_{2_{in}}V_1V_2\\
          \end{pmatrix}
          $$
          After expanding, we can have:
          $$
          \begin{pmatrix}
            \sum_{i=1}^n u_{1_{1i}}u_{2_{i1}}\cdot \sum_{j=1}^n v_{1_{1j}}v_{2_{j1}} & \sum_{i=1}^n u_{1_{1i}}u_{2_{i1}}\cdot\sum_{j=1}^n v_{1_{1j}}v_{2_{j2}} & \cdots & \sum_{i=1}^n u_{1_{1i}}u_{2_{in}}\cdot \sum_{j=1}^n v_{1_{1j}}v_{2_{jn}}\\
            \sum_{i=1}^n u_{1_{1i}}u_{2_{i1}}\cdot \sum_{j=1}^n v_{1_{2j}}v_{2_{j1}} & \sum_{i=1}^n u_{1_{1i}}u_{2_{i1}}\cdot\sum_{j=1}^n v_{1_{2j}}v_{2_{j2}} & \cdots & \sum_{i=1}^n u_{1_{1i}}u_{2_{in}}\cdot \sum_{j=1}^n v_{1_{2j}}v_{2_{jn}}\\
            \vdots & \vdots & \ddots & \vdots\\
            \sum_{i=1}^n u_{1_{ni}}u_{2_{i1}}\cdot \sum_{j=1}^n v_{1_{nj}}v_{2_{j1}} & \sum_{i=1}^n u_{1_{ni}}u_{2_{i1}}\cdot\sum_{j=1}^n v_{1_{nj}}v_{2_{j2}} & \cdots & \sum_{i=1}^n u_{1_{ni}}u_{2_{in}}\cdot \sum_{j=1}^n v_{1_{nj}}v_{2_{jn}}\\
          \end{pmatrix}
          $$
          After some transforming, we can simply find that the two expressions are the same. 
          Therefore, we proved that $(U_1\otimes V_1)\cdot(U_2\otimes V_2) = U_1U_2 \otimes V_1V_2$.
    \item Suppose that $U$ and $V$ are two vector spaces both of dimension $n$, the operator $\otimes$ can be understood as: 
          $\otimes:\ U\times V\rightarrow W$, where $W$ is also a vector space. For $u,\ u_1,\ u_2\in U$, $v,\ v_1,\ v_2\in V$, 
          we first define the following. 
          $$
          u = \begin{pmatrix} u_1\\ u_2\\ \vdots \\ u_n \end{pmatrix},\ 
          v_1 = \begin{pmatrix} v_{11}\\ v_{12}\\ \vdots \\ v_{1n} \end{pmatrix},\ 
          v_2 = \begin{pmatrix} v_{21}\\ v_{22}\\ \vdots \\ v_{2n} \end{pmatrix}\, 
          v_1 + v_2 = \begin{pmatrix} v_{11} + v_{21}\\ v_{12} + v_{22}\\ \vdots \\ v_{1n} + v_{2n} \end{pmatrix}
          $$
          So $u \otimes (v_1 + v_2)$ can be expressed as:
          \begin{align*}
              &\quad 
              \begin{pmatrix}
                  u_1(v_{11} + v_{21}) & u_1(v_{12} + v_{22}) & \cdots & u_1(v_{1n} + v_{2n})\\
                  u_2(v_{11} + v_{21}) & u_2(v_{12} + v_{22}) & \cdots & u_2(v_{1n} + v_{2n})\\
                  \vdots & \vdots & \ddots & \vdots\\
                  u_n{v_{11} + v_{21}} & u_n(v_{12} + v_{22}) & \cdots & u_n(v_{1n} + v_{2n})\\
              \end{pmatrix}\\
              &= \begin{pmatrix}
                  u_1v_{11} & u_1v_{12} & \cdots & u_1v_{1n}\\
                  u_2v_{11} & u_2v_{12} & \vdots & u_2v_{1n}\\
                  \vdots & \vdots & \ddots & \vdots\\
                  u_nv_{11} & u_nv_{12} & \cdots & u_nv_{1n} 
              \end{pmatrix} + 
              \begin{pmatrix}
                u_1v_{21} & u_1v_{22} & \cdots & u_1v_{2n}\\
                u_2v_{21} & u_2v_{22} & \vdots & u_2v_{2n}\\
                \vdots & \vdots & \ddots & \vdots\\
                u_nv_{21} & u_nv_{22} & \cdots & u_nv_{2n}
              \end{pmatrix}\\
              &= u\otimes v_1 + u\otimes v_2
          \end{align*}
          Therefore, we can have $u \otimes (v_1 + v_2) = u\otimes v_1 + u\otimes v_2$. 
          Then we want to analyze $(u_1 + u_2) \otimes v$. In the same way, we can also prove that: 
          $$(u_1 + u_2) \otimes v = (u_1\otimes v) + (u_2\otimes v)$$
          According to the above two properties being proved, we can conclude that the operate $\otimes$ is bilinear.
          
        
\end{enumerate}




\end{document}